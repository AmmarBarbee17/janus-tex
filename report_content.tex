% Report-specific content
% This file is only included when compiling in report mode

\section{Introduction}
\label{sec:introduction}
\hypertarget{intro}{}

This is the detailed introduction section of the report. Here you can provide comprehensive background, context, and motivation for your work.

It's pretty insane that this works.

In academic and technical writing, it's important to:
\begin{itemize}
    \item Establish the problem context
    \item Review relevant literature
    \item State research objectives clearly
    \item Outline the document structure
\end{itemize}

\subsection{Background}
\label{subsec:background}
\hypertarget{background}{}

Detailed background information goes here. This section can include extensive literature review, historical context, and foundational concepts necessary for understanding the work.

\subsection{Objectives}
\label{subsec:objectives}
\hypertarget{objectives}{}

State your specific objectives and goals:
\begin{enumerate}
    \item First objective with detailed explanation
    \item Second objective with context
    \item Third objective with rationale
\end{enumerate}

\section{Methodology}
\label{sec:methodology}
\hypertarget{methodology}{}

This section provides detailed methodology that would be too extensive for a presentation slide.

\subsection{Experimental Setup}
\label{subsec:setup}
\hypertarget{setup}{}

Describe your experimental or analytical setup in detail. Include:
\begin{itemize}
    \item Equipment specifications
    \item Software tools and versions
    \item Data collection procedures
    \item Quality control measures
\end{itemize}

\subsection{Data Analysis}
\label{subsec:analysis}
\hypertarget{analysis}{}

Explain your data analysis approach:

\begin{equation}
\label{eq:main}
y = f(x, \theta) = \sum_{i=1}^{n} \theta_i x^i
\end{equation}

Where $\theta$ represents the parameter vector and $x$ is the input variable. Additional mathematical derivations and proofs can be included here.

\section{Results}
\label{sec:results}
\hypertarget{results}{}

Present detailed results with comprehensive discussion.

\subsection{Quantitative Findings}
\label{subsec:quantitative}
\hypertarget{quantitative}{}

Include detailed tables, statistical analyses, and numerical results. This section can be extensive with multiple sub-analyses.

\subsection{Qualitative Observations}
\label{subsec:qualitative}
\hypertarget{qualitative}{}

Discuss qualitative findings, patterns observed, and interpretations that require extended discussion.

\section{Discussion}
\label{sec:discussion}
\hypertarget{discussion}{}

Provide in-depth discussion of results, including:
\begin{itemize}
    \item Interpretation in context of existing literature
    \item Limitations and sources of uncertainty
    \item Implications for theory and practice
    \item Unexpected findings and their significance
\end{itemize}

\section{Conclusion}
\label{sec:conclusion}
\hypertarget{conclusion}{}

Summarize the key findings and their significance. Discuss future work directions and broader impacts.

\subsection{Key Contributions}
\label{subsec:contributions}
\hypertarget{contributions}{}

List the main contributions of this work.

\subsection{Future Work}
\label{subsec:future}
\hypertarget{future}{}

Outline potential directions for future research and development.

\reportonly{
\section*{Acknowledgments}
Acknowledge funding sources, collaborators, and others who contributed to the work.
}

\reportonly{
\bibliographystyle{plain}
% \bibliography{references}  % Uncomment when you have a .bib file
}
